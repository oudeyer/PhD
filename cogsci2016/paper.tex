% 
% Annual Cognitive Science Conference
% Sample LaTeX Paper -- Proceedings Format
% 

% Original : Ashwin Ram (ashwin@cc.gatech.edu)       04/01/1994
% Modified : Johanna Moore (jmoore@cs.pitt.edu)      03/17/1995
% Modified : David Noelle (noelle@ucsd.edu)          03/15/1996
% Modified : Pat Langley (langley@cs.stanford.edu)   01/26/1997
% Latex2e corrections by Ramin Charles Nakisa        01/28/1997 
% Modified : Tina Eliassi-Rad (eliassi@cs.wisc.edu)  01/31/1998
% Modified : Trisha Yannuzzi (trisha@ircs.upenn.edu) 12/28/1999 (in process)
% Modified : Mary Ellen Foster (M.E.Foster@ed.ac.uk) 12/11/2000
% Modified : Ken Forbus                              01/23/2004
% Modified : Eli M. Silk (esilk@pitt.edu)            05/24/2005
% Modified : Niels Taatgen (taatgen@cmu.edu)         10/24/2006
% Modified : David Noelle (dnoelle@ucmerced.edu)     11/19/2014

%% Change "letterpaper" in the following line to "a4paper" if you must.

\documentclass[10pt,letterpaper]{article}

\usepackage{cogsci}
\usepackage{pslatex}
\usepackage{apacite}


\title{Curiosity-Driven Development of Tool Use Precursors}
 
\author{{\large \bf S\'ebastien Forestier (sebastien.forestier@inria.fr)} \\
	INRIA Bordeaux Sud-Ouest\\
	Bordeaux, France
  \AND {\large \bf Pierre-Yves Oudeyer (pierre-yves.oudeyer@inria.fr} \\
	INRIA Bordeaux Sud-Ouest\\
	Bordeaux, France}


\begin{document}

\maketitle


\begin{abstract}
This is the abstract.

\textbf{Keywords:} 
curiosity-driven learning; tool use; goal babbling; overlapping waves; 
\end{abstract}


\section{Introduction}

%


\section{Methods}

%


\section{Results}

%


\section{Discussion}

%


\begin{figure}[ht]
\begin{center}
\fbox{CoGNiTiVe ScIeNcE}
\end{center}
\caption{This is a figure.} 
\label{sample-figure}
\end{figure}


\section{Acknowledgments}

Place acknowledgments (including funding information) in a section at
the end of the paper.


\nocite{fabisch2014active}


\bibliographystyle{apacite}

\setlength{\bibleftmargin}{.125in}
\setlength{\bibindent}{-\bibleftmargin}

\bibliography{include/bibliography}


\end{document}
