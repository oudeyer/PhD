
%% bare_conf.tex
%% V1.3
%% 2007/01/11
%% by Michael Shell
%% See:
%% http://www.michaelshell.org/
%% for current contact information.
%%
%% This is a skeleton file demonstrating the use of IEEEtran.cls
%% (requires IEEEtran.cls version 1.7 or later) with an IEEE conference paper.
%%
%% Support sites:
%% http://www.michaelshell.org/tex/ieeetran/
%% http://www.ctan.org/tex-archive/macros/latex/contrib/IEEEtran/
%% and
%% http://www.ieee.org/

%%*************************************************************************
%% Legal Notice:
%% This code is offered as-is without any warranty either expressed or
%% implied; without even the implied warranty of MERCHANTABILITY or
%% FITNESS FOR A PARTICULAR PURPOSE! 
%% User assumes all risk.
%% In no event shall IEEE or any contributor to this code be liable for
%% any damages or losses, including, but not limited to, incidental,
%% consequential, or any other damages, resulting from the use or misuse
%% of any information contained here.
%%
%% All comments are the opinions of their respective authors and are not
%% necessarily endorsed by the IEEE.
%%
%% This work is distributed under the LaTeX Project Public License (LPPL)
%% ( http://www.latex-project.org/ ) version 1.3, and may be freely used,
%% distributed and modified. A copy of the LPPL, version 1.3, is included
%% in the base LaTeX documentation of all distributions of LaTeX released
%% 2003/12/01 or later.
%% Retain all contribution notices and credits.
%% ** Modified files should be clearly indicated as such, including  **
%% ** renaming them and changing author support contact information. **
%%
%% File list of work: IEEEtran.cls, IEEEtran_HOWTO.pdf, bare_adv.tex,
%%                    bare_conf.tex, bare_jrnl.tex, bare_jrnl_compsoc.tex
%%*************************************************************************

% *** Authors should verify (and, if needed, correct) their LaTeX system  ***
% *** with the testflow diagnostic prior to trusting their LaTeX platform ***
% *** with production work. IEEE's font choices can trigger bugs that do  ***
% *** not appear when using other class files.                            ***
% The testflow support page is at:
% http://www.michaelshell.org/tex/testflow/



% Note that the a4paper option is mainly intended so that authors in
% countries using A4 can easily print to A4 and see how their papers will
% look in print - the typesetting of the document will not typically be
% affected with changes in paper size (but the bottom and side margins will).
% Use the testflow package mentioned above to verify correct handling of
% both paper sizes by the user's LaTeX system.
%
% Also note that the "draftcls" or "draftclsnofoot", not "draft", option
% should be used if it is desired that the figures are to be displayed in
% draft mode.
%
\documentclass[conference]{include/IEEEtran}
% Add the compsoc option for Computer Society conferences.
%
% If IEEEtran.cls has not been installed into the LaTeX system files,
% manually specify the path to it like:
% \documentclass[conference]{../sty/IEEEtran}



\usepackage{cite}
\usepackage[pdftex]{graphicx}

\def\xcolorversion{2.00}
\def\xkeyvalversion{1.8}
\usepackage[version=0.96]{pgf}
\usepackage{tikz}
\usetikzlibrary{arrows,shapes,snakes,automata,backgrounds,petri}


% *** SPECIALIZED LIST PACKAGES ***
%
%\usepackage{algorithmic}
% algorithmic.sty was written by Peter Williams and Rogerio Brito.
% This package provides an algorithmic environment fo describing algorithms.
% You can use the algorithmic environment in-text or within a figure
% environment to provide for a floating algorithm. Do NOT use the algorithm
% floating environment provided by algorithm.sty (by the same authors) or
% algorithm2e.sty (by Christophe Fiorio) as IEEE does not use dedicated
% algorithm float types and packages that provide these will not provide
% correct IEEE style captions. The latest version and documentation of
% algorithmic.sty can be obtained at:
% http://www.ctan.org/tex-archive/macros/latex/contrib/algorithms/
% There is also a support site at:
% http://algorithms.berlios.de/index.html
% Also of interest may be the (relatively newer and more customizable)
% algorithmicx.sty package by Szasz Janos:
% http://www.ctan.org/tex-archive/macros/latex/contrib/algorithmicx/


\usepackage[caption=false,font=footnotesize]{subfig}


% *** FLOAT PACKAGES ***
%
\usepackage{fixltx2e}
% fixltx2e, the successor to the earlier fix2col.sty, was written by
% Frank Mittelbach and David Carlisle. This package corrects a few problems
% in the LaTeX2e kernel, the most notable of which is that in current
% LaTeX2e releases, the ordering of single and double column floats is not
% guaranteed to be preserved. Thus, an unpatched LaTeX2e can allow a
% single column figure to be placed prior to an earlier double column
% figure. The latest version and documentation can be found at:
% http://www.ctan.org/tex-archive/macros/latex/base/






% *** PDF, URL AND HYPERLINK PACKAGES ***
%
%\usepackage{url}
% url.sty was written by Donald Arseneau. It provides better support for
% handling and breaking URLs. url.sty is already installed on most LaTeX
% systems. The latest version can be obtained at:
% http://www.ctan.org/tex-archive/macros/latex/contrib/misc/
% Read the url.sty source comments for usage information. Basically,
% \url{my_url_here}.





% *** Do not adjust lengths that control margins, column widths, etc. ***
% *** Do not use packages that alter fonts (such as pslatex).         ***
% There should be no need to do such things with IEEEtran.cls V1.6 and later.
% (Unless specifically asked to do so by the journal or conference you plan
% to submit to, of course. )


% correct bad hyphenation here
\hyphenation{}


\begin{document}
%
% paper title
% can use linebreaks \\ within to get better formatting as desired
\title{Curiosity-driven Exploration of Skill Hierarchies}


% author names and affiliations
% use a multiple column layout for up to three different
% affiliations
\author{\IEEEauthorblockN{S\'ebastien Forestier}
\IEEEauthorblockA{INRIA Bordeaux Sud-Ouest\\
Bordeaux, France\\
Email: sebastien.forestier@inria.fr}
\and
\IEEEauthorblockN{Pierre-Yves Oudeyer}
\IEEEauthorblockA{INRIA Bordeaux Sud-Ouest\\
Bordeaux, France\\
Email: pierre-yves.oudeyer@inria.fr}}

% conference papers do not typically use \thanks and this command
% is locked out in conference mode. If really needed, such as for
% the acknowledgment of grants, issue a \IEEEoverridecommandlockouts
% after \documentclass

% for over three affiliations, or if they all won't fit within the width
% of the page, use this alternative format:
% 
%\author{\IEEEauthorblockN{Michael Shell\IEEEauthorrefmark{1},
%Homer Simpson\IEEEauthorrefmark{2},
%James Kirk\IEEEauthorrefmark{3}, 
%Montgomery Scott\IEEEauthorrefmark{3} and
%Eldon Tyrell\IEEEauthorrefmark{4}}
%\IEEEauthorblockA{\IEEEauthorrefmark{1}School of Electrical and Computer Engineering\\
%Georgia Institute of Technology,
%Atlanta, Georgia 30332--0250\\ Email: see http://www.michaelshell.org/contact.html}
%\IEEEauthorblockA{\IEEEauthorrefmark{2}Twentieth Century Fox, Springfield, USA\\
%Email: homer@thesimpsons.com}
%\IEEEauthorblockA{\IEEEauthorrefmark{3}Starfleet Academy, San Francisco, California 96678-2391\\
%Telephone: (800) 555--1212, Fax: (888) 555--1212}
%\IEEEauthorblockA{\IEEEauthorrefmark{4}Tyrell Inc., 123 Replicant Street, Los Angeles, California 90210--4321}}




% use for special paper notices
%\IEEEspecialpapernotice{(Invited Paper)}




% make the title area
\maketitle


\begin{abstract}
%\boldmath
The abstract goes here.
\end{abstract}
% IEEEtran.cls defaults to using nonbold math in the Abstract.
% This preserves the distinction between vectors and scalars. However,
% if the conference you are submitting to favors bold math in the abstract,
% then you can use LaTeX's standard command \boldmath at the very start
% of the abstract to achieve this. Many IEEE journals/conferences frown on
% math in the abstract anyway.

% no keywords




% For peer review papers, you can put extra information on the cover
% page as needed:
% \ifCLASSOPTIONpeerreview
% \begin{center} \bfseries EDICS Category: 3-BBND \end{center}
% \fi
%
% For peerreview papers, this IEEEtran command inserts a page break and
% creates the second title. It will be ignored for other modes.
\IEEEpeerreviewmaketitle



\section{Introduction}

	
	The study of the control of manipulation actions in humans has revealed a modular representation of actions 
	either in the cerebral cortex and in the spinal 
	cord with compositionality: an infinite number of movements can be expressed through combination of simple primitives, 
	and generalization: certain neurons (higher 
	in the hierarchy) can represent actions independently of the effectors used \cite{cangelosi2010integration}.
	
	The same idea holds for language expressiveness which is based on syntactic hierarchical combinations 
	on a vocabulary, that open infinite semantic possibilities. 
	Greenfield has also argued that this parallel between manipulation and language compositionality can be found in the human ontegenic development
	with combinatorial steps for manipulation and syntax acquired approximately at the same period and in the same order \cite{green}. 
	Also, the author explains that the development of the neural substrates for language and tool use could be an ontogenic homology as first of all
	the same neural computations for hierarchical combinations and their semantics should take place for both modalities, and furthermore 
	experiments with Broca's and Wernicke's aphasics show that hierarchical organization for language and manipulation is linked. 
	Broca's aphasics, who have less syntactic organization of speech were shown to also have problems of representation of the hierarchical organization
	of constructions with blocks, whereas Wernicke's aphasics, whose syntax is normal but speech semantics is impaired, succeed in representing such objects 
	hierarchies. 
	
	Functional MRI experiments by Higuchi et al. have shown that the human's neural substrates for tool use and language is indeed shared in 
	the dorsal BA44 Broca's area \cite{higuchi}, which gives evidence for the similar neural computations used. 
	They furthermore argue that these results supports the hypothesis that tool use have appeared first in primate evolution in F5 area,
	and then the language has developed in humans reusing part of tool use and manipulation neural substrates in human's Broca area, homolog of primate's F5.

	Like a developing child, a developmental robot will have to incrementally explore skills that add up to the hierarchy of previously learned skills 
	throughout its life, with a constraint being the cost and time of experimentation. 
	We will seek to define curiosity-driven hierarchical learning architectures 
	that could reuse the sensorimotor contingencies previously learned and to combine them 
	to explore more efficiently new complex sensorimotor models. 
		
	\subsection{Goal of the study}
			
		\begin{itemize}
			\item Exploring in a structured hierarchy is more efficient than directly from $M$ to $S$.
			\item Which task should I explore now ?
			\item How to choose between different means to explore a given space ? 
			\item How can high-level tasks guide the exploration of lower-level ones ?
			\item How can the system cope with perturbations on some of the forward models ?		
		\end{itemize}

		
	%
	
	\subsection{Related work}
		\cite{ugur2014, ugur2015}.

		Different computational models have the possibility to learn skill hierarchies. 
		In finite environments represented by a factored Markov Decision Process \cite{vig}, an intrinsic motivation towards actions 
		maximizing Dynamic Bayesian Networks' structure has been shown to allow the learning of the environment's structure.
		
		In continuous environments but with discrete actions, Metzen et al. \cite{metzen2013} use the framework of options \cite{sutton1999between} 
		to learn skill hierarchies. 
		An intrinsic motivation rewards positively the novelty of the states encountered and negatively the prediction error of the learned skill model.
		
		The model from Fabisch et al. \cite{fabisch2014active} learns in a setting with a discrete task space (called contexts).
		It uses an intrinsic motivation for learning progress, and a Multi-Armed Bandit algorithm (D-UCB) to choose on which context the agent should train for.
		The Upper Confidence Bound algorithm chooses between contexts given their estimated learning progress and the uncertainty of these estimations
		by picking the context with the maximum upper confidence bound.
		In other words, it maximizes the expected reward plus something related to the uncertainty associated with it, selecting either contexts with 
		certain high rewards or ones with uncertain poor reward.
		This algorithm embeds directly a solution the exploration-exploitation trade-off problem as it represents the exploitation 
		of knowledge by the expected progress and the exploration of other solutions by the uncertainty bonus. 
		This algorithm supposes a stationary learning progress on each context so the authors use an adaptation 
		(D-UCB, \cite{kocsis2006discounted}) to encompass non-stationary learning progress.
		% Sliding Window UCB: \cite{garivier2011upper}
		
		In a fully continuous setting, Mugan et al. \cite{mugan} have developed an algorithm that first learns a qualitative representation of environment states
		and actions in order to then learn the structure of Dynamic Bayesian Networks representing the temporal contingencies of those states and actions.
		In order to choose which action to practice, the authors use the IAC \cite{oudeyer_intrinsic_2007} where the agent 
		is intrinsically motivated to choose actions that are estimated to yield high prediction error progress.
		
		Here, we will rely more specifically on the SAGG-RIAC architecture \cite{baranes_active_2013}.
		This architecture learns a single mapping between continuous motor and sensori (or task) 
		spaces with a competence-based intrinsic motivation. 
		In our hierarchy of sensorimotor models, each model will be explored using the SAGG-RIAC procedure, but it could be replaced by another one 
		without changing the mechanisms to learn the hierarchy that will be assessed.

	%
%


	
\section{Methods}
	
	\subsection{Algorithms}
	
		\paragraph{}
		Different module implementations (implemented):
		\begin{itemize}
			\item Motor Babbling,
			\item SAGG-Random.
		\end{itemize}
		
		\paragraph{}
		Different types of Sensorimotor models (implemented):
		\begin{itemize}
			\item NN or LWLR,
			\item NSLWLR to handle non stationary forward models (See Section \ref{NSLWLR}), or NSNN as NN with max gaussian\_kernel(distance) * gaussian\_kernel(time\_distance).
		\end{itemize}
		
		\paragraph{}
		Different possibilities to handle exploration in the hierarchy (implemented):
		\begin{itemize}
			\item MAB on all modules (greedy, softmax),
			\item MAB biaised on lower modules (see Sec. \ref{NSLWLR}),
			\item ZPDES (see Sec. \ref{zpdes}).
		\end{itemize}
		
		\paragraph{}
		Different types of Top-Down Drive:
		\begin{itemize}
			\item Just add noise to motor command of each module (pb: interferes with competence and progress estimation)  (implemented),
			\item Exploration budget around goals asked by higher models (explonential in $levels^{n+1}$ so even $n=1$ is not feasible in general but maybe in practice)  (implemented),
			\item Exploration budget only for the modules one layer below the babbling module (not implemented).
			\item Balance self-computed interest with Top-Down interest (implemented, based on the density of top-down goals).
		\end{itemize}
		
	
	%
	
	\subsection{Hierarchies}
	
		See Fig. \ref{Hierarchies} for exemples of hierarchies.
	
		\begin{figure}[!t]
			\centerline{
			\subfloat[]{
% H2
\begin{tikzpicture}[node distance=1cm,>=stealth',bend angle=45,auto]

	\tikzstyle{dom}   = [ellipse, thick, draw=black!75, fill=blue!20,  minimum height=4mm, minimum width=4mm]
	\tikzstyle{prim dom}   = [dom,  minimum height=5mm, minimum width=5mm,  fill=green!20]
	\tikzstyle{mod} = [rectangle, thick, draw=black!75, fill=black!20, minimum size=3mm]

	\begin{scope}
		\scriptsize
		\node [prim dom, label=below:$12$D] (pd1) {$Arm$};

		\node [mod, label=above:$Model_1$] (m1) [right of=pd1, xshift=-0.1cm] {$1$}
		edge [pre]                  (pd1);

		\node [dom, label=below:$9$D] (d1) [right of=m1] {$Hand$}
		edge [pre]                (m1);
		
		\node [mod] (m3) [right of=d1, xshift=-0.1cm,yshift=0.4cm] {$2$}
		edge [pre]                  (d1);
		
		\node [dom, label=below:$6$D] (d3) [right of=m3, xshift=-0.1cm,yshift=0.4cm] {$Stick_1$}
		edge [pre]                (m3);
		
		\node [mod] (m6) [right of=d1, xshift=-0.1cm,yshift=-0.4cm] {$5$}
		edge [pre]                  (d1);
		
		\node [dom, label=above:$6$D] (d6) [right of=m6, xshift=-0.1cm,yshift=-0.4cm] {$Stick_2$}
		edge [pre]                (m6);
		
		\node [mod] (m7) [right of=d6,yshift=0.2cm] {$6$}
		edge [pre]                  (d6);
		
		\node [mod] (m4) [right of=d3,yshift=-0.2cm] {$3$}
		edge [pre]                  (d3);
		
		\node [dom, label=below:$2$D] (d4) [right of=m4, xshift=-0.3cm,yshift=-0.6cm] {$Object$}
		edge [pre]                (m4)
		edge [pre]                (m7);
		
		\node [mod] (m5) [right of=d4] {$4$}
		edge [pre]                  (d4);
		
		\node [dom, label=below:$2$D] (d5) [right of=m5] {$Boxes$}
		edge [pre]                (m5);
		
		
	\end{scope}
	\begin{pgfonlayer}{background}
		\filldraw [line width=2mm,black!10]
		(d3.north  -| d5.east)  rectangle (d6.south  -| pd1.west);
	\end{pgfonlayer}
\end{tikzpicture}
}
			\subfloat[]{
% H2
\begin{tikzpicture}[node distance=1.cm,>=stealth',bend angle=45,auto]

	\tikzstyle{dom}   = [ellipse, thick, draw=black!75, fill=blue!20,  minimum height=5mm, minimum width=5mm]
	\tikzstyle{prim dom}   = [dom,  minimum height=5mm, minimum width=5mm,  fill=green!20]
	\tikzstyle{mod} = [rectangle, thick, draw=black!75, fill=black!20, minimum size=4mm]

	\begin{scope}
		\scriptsize
		\node [prim dom] (pd1) {$Arm_1$};
		
		\node [mod] (m1) [above of=pd1] {}
		edge [pre]                  (pd1);

		\node [dom] (d1) [above of=m1] {$Hand_1$}
		edge [pre]                (m1);
		
		\node [mod] (m2) [above of=d1, xshift=-1cm] {}
		edge [pre]                  (d1);
		
		\node [mod] (m3) [right of=m2, xshift=1cm] {}
		edge [pre]                  (d1);
		
		\node [dom] (d2) [above of=m2] {$Tool_1$}
		edge [pre]                (m2);
		
		\node [dom] (d3) [above of=m3] {$Tool_2$}
		edge [pre]                (m3);
		
		\node [mod] (m4) [above of=d2] {}
		edge [pre]                  (d2);
		
		\node [mod] (m5) [above of=d3] {}
		edge [pre]                  (d3);
		
		\node [dom] (d4) [above of=m4, xshift=1cm] {$Obj_1$}
		edge [pre]                (m4)
		edge [pre]                (m5);
		
		
	\end{scope}
	\begin{pgfonlayer}{background}
		\filldraw [line width=4mm,join=round,black!10]
		(d4.north  -| d3.east)  rectangle (pd1.south  -| d2.west);
	\end{pgfonlayer}
\end{tikzpicture}
}}
			\caption{Exemple of hierarchies.}
			\label{Hierarchies}					
		\end{figure}
	%
	
	
	\subsection{Environment}
	
		\paragraph{}
		Different variables to test: motor and sensory dimensionalities, type of functions (linear, cosine, geometric arm, non-continuous interaction with objects), noise.
		
		\paragraph{Possible arm models}
		\begin{itemize}
			\item $n$ DOFs 2D geometric arm
		\end{itemize}
		
		\paragraph{Possible tool models}
		\begin{itemize}
			\item static piano
			\item stick 
		\end{itemize}
		
		\paragraph{Possible object models}
		\begin{itemize}
			\item static object with features depending on the piano
			\item dynamic object
		\end{itemize}
		
	
	%
	

	\subsection{Experiments}

			% Experiment 1
			\paragraph{Experiment 1: Exploring in a structured hierarchy is more efficient than directly from $M$ to $S$.}				
			\begin{itemize}
				\item Idea: to compare our simplest algorithm (explore the module with higher progress) in a realistic setting (Hierarchy (a) of Fig. \ref{Hierarchies}) to the control condition where a sensorimotor model is learned
						directly from the whole motor space $M$ to the whole sensori space $S$.
				
				\item Conditions: Hierarchy (a) vs $M \rightarrow S$, Motor Babbling vs SAGG-Random
				
				\item Features: MAB on all modules, NN, No TDD.
				
				\item Measures: exploration of intermediate spaces (hands, tools), exploration of top spaces (objects). Competence to reach random goals in reachable parts of intermediate and top spaces. 
						Statistics on multiple runs to see regularity/diversity in developmental trajectories.
			\end{itemize}
			
			% Experiment 2
			\paragraph{Experiment 2: Which task should I explore now ?}
			\begin{itemize}
				\item Idea: to compare the different possibilities to choose the module to explore in the hierarchy: maximizing the progress, maximizing with a bias towards lower-level modules, or use ZPDES,
						with the same hierarchy (a) of Fig. \ref{Hierarchies}.
				
				\item Conditions: Random module, MAB on all modules, MAB with bias, ZPDES.
				
				\item Features: Hierarchy (a), SAGG-Random, NN, No TDD.
				
				\item Measures: exploration of intermediate spaces (hands, tools), exploration of top spaces (objects). Competence to reach random goals in reachable parts of intermediate and top spaces. 
						Statistics on multiple runs to see regularity/diversity in developmental trajectories.
			\end{itemize}
			
			% Experiment 3
			\paragraph{Experiment 3: How to choose between different means to explore a given space ?}
			\begin{itemize}
				\item Idea: to explain how we can choose between different means (e.g. different tools) using the one with the maximal competence, or maximal progress.
						We can use the hierarchy (b) of Fig. \ref{Hierarchies}, in order to have 2 different tools to move the object.
				
				\item Conditions: maximize competence vs progress, noise on each tool, reachability (e.g. size) of each tool.
				
				\item Features: Hierarchy (b), MAB on all modules, SAGG-Random, NN, No TDD.
				
				\item Measures: exploration of intermediate spaces (hand, tools), exploration of top spaces (object). Competence to reach random goals in reachable parts of intermediate and top spaces. 
						Statistics on multiple runs to see regularity/diversity in developmental trajectories.
			\end{itemize}
			
			% Experiment 4
			\paragraph{Experiment 4: How can high-level tasks guide the exploration of lower-level ones ?}
			\begin{itemize}
				\item Idea: to compare different possibilities of Top-Down Guidance, with hierarchy (c) of Fig. \ref{Hierarchies} in order to have TD guidance at different levels: 
						arm with one higher model, hand with 2 higher models, tool1 with one higher model, or tool2 with 2 higher models.
				
				\item Conditions: No TDD, Just add noise to motor command of each module (pb: interferes with competence estimation) 
						vs explore $n$ points and returns the best (warning: exponential)
						vs only one layer below the babbling module explores $n$ points.
				
				\item Features: Hierarchy (c), MAB on all modules with bias (or no bias?), SAGG-Random, NN.
				
				\item Measures: exploration of intermediate spaces (hands, tools), exploration of top spaces (objects). Competence to reach random goals in reachable parts of intermediate and top spaces. 
						Statistics on multiple runs to see regularity/diversity in developmental trajectories.
			\end{itemize}
			
			% Experiment 5
			\paragraph{Experiment 5: How can the system cope with perturbations on some of the forward models ?}
			\begin{itemize}
				\item Idea: to apply perturbations to one of the possible forward models, either blocking, shifting or randomizing one dimension. 
						We can use the hierarchy (d) of Fig. \ref{Hierarchies} in order to see an adaptation at 2 levels: the use of one hand to use one tool, the use of both tools to move the object even if only one is perturbated.
				
				\item Conditions: No perturbations, which model is perturbated (arm, tool), type of perturbation (blocking, shifting, random).
				
				\item Features: Hierarchy (d), MAB on all modules, SAGG-Random, NN, best TDD.
				
				\item Measures: exploration of intermediate spaces (hands, tools) and top spaces (objects) before and after perturbations. 
						Competence to reach random goals in reachable parts of intermediate and top spaces before and after perturbations. 
						Statistics on multiple runs to see regularity/diversity in developmental trajectories.
			\end{itemize}

%


\section{Results}

%


\section{Discussion}

%


% An example of a floating figure using the graphicx package.
% Note that \label must occur AFTER (or within) \caption.
% For figures, \caption should occur after the \includegraphics.
% Note that IEEEtran v1.7 and later has special internal code that
% is designed to preserve the operation of \label within \caption
% even when the captionsoff option is in effect. However, because
% of issues like this, it may be the safest practice to put all your
% \label just after \caption rather than within \caption{}.
%
% Reminder: the "draftcls" or "draftclsnofoot", not "draft", class
% option should be used if it is desired that the figures are to be
% displayed while in draft mode.
%
%\begin{figure}[!t]
%\centering
%\includegraphics[width=2.5in]{myfigure}
% where an .eps filename suffix will be assumed under latex, 
% and a .pdf suffix will be assumed for pdflatex; or what has been declared
% via \DeclareGraphicsExtensions.
%\caption{Simulation Results}
%\label{fig_sim}
%\end{figure}

% Note that IEEE typically puts floats only at the top, even when this
% results in a large percentage of a column being occupied by floats.


% An example of a double column floating figure using two subfigures.
% (The subfig.sty package must be loaded for this to work.)
% The subfigure \label commands are set within each subfloat command, the
% \label for the overall figure must come after \caption.
% \hfil must be used as a separator to get equal spacing.
% The subfigure.sty package works much the same way, except \subfigure is
% used instead of \subfloat.
%
%\begin{figure*}[!t]
%\centerline{\subfloat[Case I]\includegraphics[width=2.5in]{subfigcase1}%
%\label{fig_first_case}}
%\hfil
%\subfloat[Case II]{\includegraphics[width=2.5in]{subfigcase2}%
%\label{fig_second_case}}}
%\caption{Simulation results}
%\label{fig_sim}
%\end{figure*}
%
% Note that often IEEE papers with subfigures do not employ subfigure
% captions (using the optional argument to \subfloat), but instead will
% reference/describe all of them (a), (b), etc., within the main caption.


% An example of a floating table. Note that, for IEEE style tables, the 
% \caption command should come BEFORE the table. Table text will default to
% \footnotesize as IEEE normally uses this smaller font for tables.
% The \label must come after \caption as always.
%
%\begin{table}[!t]
%% increase table row spacing, adjust to taste
%\renewcommand{\arraystretch}{1.3}
% if using array.sty, it might be a good idea to tweak the value of
% \extrarowheight as needed to properly center the text within the cells
%\caption{An Example of a Table}
%\label{table_example}
%\centering
%% Some packages, such as MDW tools, offer better commands for making tables
%% than the plain LaTeX2e tabular which is used here.
%\begin{tabular}{|c||c|}
%\hline
%One & Two\\
%\hline
%Three & Four\\
%\hline
%\end{tabular}
%\end{table}


% Note that IEEE does not put floats in the very first column - or typically
% anywhere on the first page for that matter. Also, in-text middle ("here")
% positioning is not used. Most IEEE journals/conferences use top floats
% exclusively. Note that, LaTeX2e, unlike IEEE journals/conferences, places
% footnotes above bottom floats. This can be corrected via the \fnbelowfloat
% command of the stfloats package.



\section{Conclusion}
The conclusion goes here.




% conference papers do not normally have an appendix


% use section* for acknowledgement
\section*{Acknowledgment}


The authors would like to thank...





% trigger a \newpage just before the given reference
% number - used to balance the columns on the last page
% adjust value as needed - may need to be readjusted if
% the document is modified later
%\IEEEtriggeratref{8}
% The "triggered" command can be changed if desired:
%\IEEEtriggercmd{\enlargethispage{-5in}}

% references section

% can use a bibliography generated by BibTeX as a .bbl file
% BibTeX documentation can be easily obtained at:
% http://www.ctan.org/tex-archive/biblio/bibtex/contrib/doc/
% The IEEEtran BibTeX style support page is at:
% http://www.michaelshell.org/tex/ieeetran/bibtex/
\bibliographystyle{include/IEEEtran}
% argument is your BibTeX string definitions and bibliography database(s)
%\bibliography{IEEEabrv,../bib/paper}
%
% <OR> manually copy in the resultant .bbl file
% set second argument of \begin to the number of references
% (used to reserve space for the reference number labels box)

%\begin{thebibliography}{1}

%\bibitem{IEEEhowto:kopka}
%H.~Kopka and P.~W. Daly, \emph{A Guide to \LaTeX}, 3rd~ed.\hskip 1em plus
  %0.5em minus 0.4em\relax Harlow, England: Addison-Wesley, 1999.

%\end{thebibliography}

\bibliography{./include/bibliography}



% that's all folks
\end{document}


